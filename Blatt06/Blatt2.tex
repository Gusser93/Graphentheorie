\documentclass[a4paper,11pt,twoside]{scrartcl}
\usepackage[T1]{fontenc}
\usepackage{subcaption}
\usepackage[utf8]{inputenc}
\usepackage{ngerman, eucal, mathrsfs, amsfonts, bbm, amsmath, amssymb, stmaryrd,graphicx, array, geometry, color, wrapfig, float, hyperref, epstopdf,gensymb, subcaption, extarrows}
\geometry{left=25mm, right=15mm, bottom=25mm}
\setlength{\parindent}{0em} 
\setlength{\headheight}{0em} 
\title{Graphenalgorithmen\\ Blatt 6}
\author{Markus Vieth\and Christian Stricker}
\date{\today}
\usepackage{listings, textcomp}
\usepackage[usenames,dvipsnames,svgnames,table]{xcolor}


\definecolor{Code}{rgb}{0,0,0}
\definecolor{Keywords}{rgb}{0,0,255}
\definecolor{Strings}{rgb}{255,0,0}
\colorlet{Comments}{Green}
\colorlet{Numbers}{blue}

%%%%%%%%%%%
%Mache Integer farbig
%%%%%%%%%%%

\makeatletter

\newif\iffirstchar\firstchartrue
\newif\ifstartedbyadigit

\newcommand\processletter
{%
	\ifnum\lst@mode=\lst@Pmode%
	\iffirstchar%
	\global\startedbyadigitfalse%
	\fi
	\global\firstcharfalse%
	\fi
}

\newcommand\processdigit
{%
	\ifnum\lst@mode=\lst@Pmode%
	\iffirstchar%
	\global\startedbyadigittrue%
	\fi
	\global\firstcharfalse%
	\fi
}

\lst@AddToHook{Output}%
{%
	\ifstartedbyadigit%
	\def\lst@thestyle{\color{Numbers}}%
	\fi
	\global\firstchartrue%
	\global\startedbyadigitfalse%
}

\newtoks\jubo@toks
\jubo@toks={
	language=C,
	commentstyle=\color{Comments}\slshape,
	stringstyle=\color{Strings},
	keywordstyle={\color{Keywords}\bfseries},
	alsoletter=0123456789,
	SelectCharTable=%
}
\def\add@savedef#1#2{%
	\begingroup\lccode`?=#1\relax
	\lowercase{\endgroup
		\edef\@temp{%
			\noexpand\lst@DefSaveDef{\number#1}%
			\expandafter\noexpand\csname lsts@?\endcsname{%
				\expandafter\noexpand\csname lsts@?\endcsname\noexpand#2}%
		}}%
		\jubo@toks=\expandafter{\the\expandafter\jubo@toks\@temp}%
	}
	\count@=`0
	\loop
	\add@savedef\count@\processdigit
	\ifnum\count@<`9
	\advance\count@\@ne
	\repeat
	\count@=`A
	\loop
	\add@savedef\count@\processletter
	\ifnum\count@<`Z
	\advance\count@\@ne
	\repeat
	\count@=`a
	\loop
	\add@savedef\count@\processletter
	\ifnum\count@<`z
	\advance\count@\@ne
	\repeat
	%\showthe\jubo@toks % for debugging
	\begingroup\edef\x{\endgroup
		\noexpand\lstdefinestyle{pseudo}{\the\jubo@toks}
	}\x
	
	\makeatother
%%%%%%%%%%
%Ende
%%%%%%%%%%



\lstset{
	literate={ö}{{\"o}}1
	{ä}{{\"a}}1
	{ü}{{\"u}}1
	{ß}{{\ss}}1
	{/pi}{{$\Pi$}}1
	{/inf}{{$\infty$}}1
	{/eIn}{{$\in$}}1
	{/cup}{{$\cup$}}1
	{/leer}{{$\emptyset$}}1
	{<=}{{$\leq$}}1
	{>=}{{$\geq$}}1
	{→}{{$\rightarrow$}}1
	{⊈}{{$\not\subseteq$}}1
	{⊆}{{$\subseteq$}}1
	{∉}{{$\notin$}}1
	{∈}{{$\in$}}1
	{⇒}{{$\Rightarrow$}}1
	{∪}{{$\cup$}}1
	{α}{{$\alpha$}}1
	{∅}{{$\emptyset$}}1	
}


\lstset{
	numberstyle=\tiny,
	stepnumber=1,
	numbersep=10pt,
	xleftmargin=15pt,
	breaklines=true,
	numberblanklines=false,
	showstringspaces=false,
	flexiblecolumns=true,
	mathescape=true,
	tabsize=4,
	captionpos=b,
	numbers=left,
	commentstyle=\color{Green},
	numberstyle=\color{gray},
	keywordstyle=\color{blue} \textbf,%otherkeywords={xdata},
	keywords=[2]{xdata},
	keywordstyle=[2]\color{red}\textbf,
	identifierstyle=\color{black},
	stringstyle=\color{red}\ttfamily,
	basicstyle = \ttfamily \color{black} \footnotesize,
	inputencoding=utf8,
	emph=[1]%
	{%
		infinity,
	}, 
	emphstyle=[1]{\color{blue}},
	emph=[2]%
	{%
		forall,
		while,
		if,
		else,
		for,
		return,
		new,
		NULL,
		null,
		int, 
		double, 
		float,
		class,
		void,
		false, 
		true,
		FALSE,
		TRUE,
	}, 
	emphstyle=[2]{\color{Magenta}},
	emph=[3]{b0, b1, n0, n1},
	emphstyle=[3]{\color{black}}
}
\usepackage{float}
\usepackage[section]{placeins}
\usepackage{epstopdf}
\usepackage{wrapfig}
\usepackage{caption}
\usepackage{subcaption}
\usepackage{graphicx}
\usepackage{pgfplots}
\usetikzlibrary{plotmarks}
\usetikzlibrary{patterns}
\usetikzlibrary{decorations.pathmorphing}
\newcommand{\coloredcircled}[3][black]{{\large \Large\color{#2}\textcircled {{\small\color{#1}#3}}}}% Circlecolor, Textcolor, text
\newcommand{\ddvec}[2]{\begin{pmatrix}#1\\#2\end{pmatrix}}
\newcommand{\dddvec}[3]{\begin{pmatrix}#1\\#2\\#3\end{pmatrix}}
\newcommand{\longvec}[1]{\overset{\longrightarrow}{#1}}
\newcommand{\eunorm}[1]{\left\lVert#1\right\rVert_2}
\newcommand{\scalar}[2]{\left<#1,#2\right>}\newcommand{\cor}[1]{\textcolor{red}{\textit{#1}}}
\newcommand{\qed}{%
	\begin{flushright}
		q.e.d.
	\end{flushright}%
	}
\begin{document}
\maketitle
\cleardoublepage
\pagestyle{myheadings}
\markboth{Markus Vieth, Christian Stricker}{Markus Vieth, Christian Stricker}

\newpage
\section{Aufgabe 1}
\paragraph{Pseudocode}
Sei \texttt{E} das Ergebnis des Monte Carlo Algorithmus, \texttt{I} die Eingabe und \texttt{boolean test(Ergebnis E)} der Algorithmus, welcher das Ergebnis überprüft.
\begin{lstlisting}
Ergebnis E;
do {
E = MonteCarloAlgorithmus(I); 
}while(!test(E));
return E;
\end{lstlisting}
\paragraph{Laufzeit}
Sei $q(n)=1-p(n)$ und $P(X = i) := q(n)^{i-1}\cdot p(n)$ die Wahrscheinlichkeit, dass der Monte Carlo Algorithmus nach $i$ Durchläufen das richtige Ergebnis liefert (also zuerst $i-1$ falsche und dann ein richtiges). Dann beträgt die zu erwartende Anzahl an Durchläufen:
\[ E(X)=\sum_{i=1}^{\infty} i\cdot q(n)^{i-1}\cdot p(n)=\footnote{$\sum_{i=1}^{\infty} i\cdot q(n)^{i-1}\cdot p(n)$ ist absolut konvergent, wenn $0<p(n)<1$, kann per Quotientenkriterium gezeigt werden}p(n)\cdot\sum_{i=1}^{\infty} i\cdot q(n)^{i-1}=\footnote{Ableitung der geometrischen Reihe} p(n)\cdot\frac{1}{(1-q(n))^2}=\frac{p(n)}{(1-1+p(n))^2}=\frac{1}{p(n)}  \]
Sei $T_{LV}(n)$ die Laufzeit des oben beschriebenen Algorithmus:
\[ \Rightarrow T_{LV}(n)=E(X)\cdot (T(n)+t(n))=\frac{T(n)+t(n)}{p(n)}  \]

Sei $X$ die Anzahl der Durchläufe
$ X \sim \text{geom}_p \Rightarrow E[X] = \frac{1}{p(n)} $
\[ P(X=k)=(1-p)^{k-1}\cdot p \]
\[ E[X]  = \sum_{k = 1}^{\infty}(1-p)^{k-1}pk = p\sum_{k = 1}^{\infty}(1-p)^{k-1}k = p \sum_{n=0}^{\infty}\frac{-d}{dp}(1-p)^k = -p\frac{d}{dp} \sum_{n=0}^{\infty}(1-p)^k  = -p\frac{d}{dp}\frac{1}{p} = \frac{1}{p} \]
\begin{flushright}
	q.e.d.
\end{flushright}
\section{Aufgabe 2}
\subsection{Pseudocode}
\begin{lstlisting}
procedure contract($G=(V,E),t$):
	while $|V| > t$
		choose $e \in E$ uniforly at random
		$ G \leftarrow E\setminus\{e\}$
	return $G$
	
findMin3Cut($G=(V,E)$)
	(V,E) = contract(G,3)
	return E

min($E_1, E_2$)
	return E_1 < E_2 ? E_1 : E_2
	
betterFindMin3Cut($G=(V,E)$)
	return min(findMin3Cut(G), findMin3Cut(G))
\end{lstlisting}
\subsection{Max Anzahl an Kanten in einem 3-Cut}
z.z. max Anzahl ist $ \left( 1-\left( 1-\frac{2}{n} \right)\left( 1-\frac{2}{n-1} \right) \right)m$
\paragraph{Beweis}
Angenommen wir wählen 2 Knoten zufällig gleichverteilt. Wir nehmen nur an das die beiden Knoten je einen Teil des 3-Cuts ausmachen und die restlichen Knoten den 3. Teil. Eine Kante gehört zum 3-Cut, wenn sie in verschiedenen so definierten Partitionen endet. Die Wahrscheinlichkeit, dass eine Kante nun zum 3-Cut gehört ist die Wahrscheinlichkeit, dass einer der Endpunkte in einer der 2 vorher gewählten Knoten ist, also:
\[ P = 1-\left( 1- \frac{2}{n} \right)\left( 1-\frac{2}{n-1} \right) \]
$f$ sein nun die erwartete Zahl an Kanten im 3-Cut und $m$ die Anzahl an Kanten im Graphen, dann gilt:
\[ E[f] = \left( 1-\left( 1-\frac{2}{n} \right)\left( 1-\frac{2}{n-1} \right) \right)m \]
Da es nicht weniger Kanten in einem 3-Cut geben kann als im minimalen 3-Cut, ist auch $E[f]$ größergleich einem minimalen 3-Cut
\subsection{Wahrscheinlichkeit zum finden eines min 3-Cuts mit einer Iteration}
parallel zur Vorlesung gilt:
\begin{align*}
P &= \prod_{i = 0}^{n-4}\left( \left( 1-\frac{2}{i} \right)\left( 1-\frac{2}{n-i-1} \right) \right)\\
&= \prod_{i = 0}^{n-4} \left( 1-\frac{2}{n-i} \right)\prod_{i = 0}^{n-4}\left( 1-\frac{2}{n-i-1} \right)\\
&=\binom{3}{2}\binom{n}{2}^{-1}\binom{n-1}{2}^{-1}
\end{align*}
\section{Aufgabe 3}
\subsection{Wahrscheinlichkeit für 1 Iteration falsch}
\[ P(n) = \left( 1-\left( 1- \left( 1- P\left(\lceil1+\frac{n}{\sqrt{2}} \rceil\right) \right)^2 \right) \right)\]
\subsection{Wahrscheinlichkeit für 2 Iterationen richtig}

\[ P(n) = 1- \left( 1-\left( 1- \left[ 1- P\left(\left\lceil1+\frac{n}{\sqrt{2}} \right\rceil\right) \right]^2 \right) \right)^2 = 1-\left[ 1- P\left(\left\lceil1+\frac{n}{\sqrt{2}} \right\rceil\right) \right]^4
\]
\section{Aufgabe 4}
Wahrscheinlichkeit zum finden eines Min-Cut:
\[ P(n) = \prod_{i=1}^{n-2} \frac{n-i-1}{n-i+1} = \frac{\prod_{i=1}^{n-2} n-i-1}{\prod_{j=1}^{n-2}n - j + 1} = \frac{\overset{n-3}{\overbrace{2}} \cdot \overset{n-2}{\overbrace{1}}}{(n)(n-1)} \cdot \frac{\prod_{i = 1}^{n-4} n - i - 1}{\prod_{j = 3}^{n - 2} n - j + 1}\]
\[ = \frac{2}{n(n-1)} \cdot \frac{\prod_{i = 1}^{n-4} n - i - 1}{\prod_{i = 1}^{n - 4} n - (i + 2) + 1}= \frac{2}{n(n-1)} \cdot \prod_{i = 1}^{n-4}\frac{ n - i - 1}{ n - i - 1} = \frac{2}{n(n-1)}\cdot 1 \]
Seien $C_1,\ldots,C_k$ Min-Cuts in $G$. Sei $D_i$ das Event, dass der Algorithmus $C_i$ findet. Da alle  $D_i$ unabhängig sind, gilt
\[ \sum_{i=1}^{k}P(D_i) \leq 1 \]
mit der Rechnung von oben gilt:
\[ \Rightarrow P(D_i) = P(n) \Rightarrow \sum_{i=1}^{k} \frac{2}{n(n-1)} \leq 1 \Rightarrow k = \frac{n(n-1)}{2} \]
\qed
\pagebreak
\part{Korrektur}
\section{2}
\emph{Multigraph}
\[ deg(u) =: d(u) \]
\[ d(u,v) = \text{Anzahl der Kanten zwischen u und v} \]
\[ E(U,V) = \text{alle Kanten die U und V verbinden} \]
\[ n(n-1)k \leq \sum_{u\neq v} d(u)+d(v) + d(u,v)\]
\[ = (2(n-1)\underset{2n}{\underbrace{\sum_{u} d(u)}}) - 2m \]
\[ = (2n -2)2m - 2m \]
\[ = (4n - 6)m \]
\[ \Rightarrow m \geq \frac{n(n-1)}{kn-6}k \]
\[ P(E) \geq 1 - \frac{4n - 6}{n(n-1)} \geq 1 -\frac{4}{n} = \frac{n - 4}{n} \]
\[ \text{Allg} P(E_i | \bigcap_{i > j} E_j) \geq 1 - \frac{4}{n-i+1} = \frac{n - i - 3}{n - i + 1} \]
\[ P(\text{Erfolgreich}) = P(\bigcap_{i=1}^{n-3} (E_i | \bigcap_{j<i}E_j)) \]
\[ =\prod_{i=1}^{n-3}P(E_i|\bigcap_{j>i}E_j) \]
\[ \geq \prod \frac{n - i - 3}{n - i + 1} \]
\[ = \Omega(n^{-4}) \]
\section{3}
\[ P[\text{falsch}] \leq \left( 1 - \frac{2}{n^2} \right)^x \]
\[ P[\text{treffer}] \leq 1 - P[falsch] \geq 1 - (1 - \frac{2}{n^2})^2 \geq \frac{4n^2 - 4}{n^4}  \]
\[ 2(n-2) = 2n - 4 \]
\section{4}
Ereignisse sind disjunkt nicht unabhängig.
\end{document}